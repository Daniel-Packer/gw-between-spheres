\documentclass{article}

\usepackage{xcolor}

\begin{document}
\section{Experiments}
The explicit computations of the $(4, 2)$-Gromov-Wasserstein distance between Euclidean spheres provides a helpful tool for benchmarking common optimal transport packages.
The authors are aware of two well-known and commonly used python implementations of optimal transport solvers: Python Optimal Transport (POT) \textcolor{blue}{CITE} and Optimal Transport Tools (OTT) \textcolor{blue}{CITE}.
These package implement two of the common methods for computing the Gromov-Wasserstein distance--either with entropic regularization (as with OTT) or without (as implemented in POT).\footnote{In fact, POT provides implementations of both solving techniques, but we include OTT, which only provides the regularized solver, since it is faster \textcolor{blue}{CITE} because of its JAX implementation.}

The goal of this section is to benchmark various sampling methods and the number of samples required to obtain accurate estimates of the Gromov-Wasserstein distance while also understanding the accuracy of the various solvers.

We run two sorts of experiments.
First, we will examine how the number of samples relates to the choice of a regularized or non-regularized solver.
Second, we will fix the number of samples and vary the dimensionality of the spheres.

\subsection{Varied points experiment}




\end{document}